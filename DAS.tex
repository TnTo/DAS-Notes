% !TeX spellcheck = en_GB
\documentclass[a4paper, headings=standardclasses]{scrartcl}

\usepackage[margin=1in]{geometry}
\usepackage{authblk}
\renewcommand{\Affilfont}{\small}
\usepackage[style=authoryear, backend=biber, sorting=nyt, useprefix=true]{biblatex}
\usepackage[autostyle=false, style=english]{csquotes}
\MakeOuterQuote{"}
\usepackage[english]{babel}
\usepackage{amsmath}
\usepackage{framed}
\usepackage[hidelinks]{hyperref}
\usepackage{booktabs}
\usepackage{enumitem}
\usepackage{cleveref}

\addbibresource{DAS.bib}

\newenvironment{enh}[1][]{\begin{framed}\noindent\textbf{Enhancement: #1}\par}{\end{framed}}

\newcommand{\todo}[1]{\par \textbf{ToDo:} #1}

\newlist{steps}{enumerate}{10}
\setlist[steps]{label*=\arabic*}

\crefname{stepsi}{step}{steps}
\Crefname{stepsi}{Step}{Steps}


%opening
\title{Dreaming of Artificial Society}
\subtitle{Working Notes on an AB-SFC PK Macroeconomic Model}
\author{Michele Ciruzzi\thanks{mciruzzi@uninsubria.it - https://orcid.org/0000-0003-1485-1204}}

\begin{document}
	
	\maketitle

	\tableofcontents

\section{Introduction}
\subsection{Aims}
The aim of this model is to highlight the macroeconomic and distributional effects of some welfare policies.  
The focus is put in particular on some (recent) policies yet unapplied in real world as Universal Basic Income, Job Guarantee schemes or the presence of only cooperative firms.

\section{General Hypothesis}
\subsection{Time}
The simulation's time-span has to be long enough to observe the effects of introducing a policy. But, it is unreasonable to prolong the simulation over 5-10 years after the policy's introduction because, in any real world context, a government is able to tune or revert the policy afterwards.

Choosing a time-frame of 7 to 15 years (with the policy's introduction after 2 to 5 years), allows simulating the model with a higher frequency to smooth out behaviours: considering 12 months per year and 4 weeks per month, simulating at week-scale require 48 ticks per year, and so 336 to 720 ticks (or alternatively 84 to 180 using monthly time-scale).

\subsection{Close Economy}
The assumption of a close economy strongly reduce the complexity of the model, but prevent to observe some economic phenomena like an export-led growth (such as Italy or Germany) or the offshoring labour-intensive productions.
Nevertheless, this is a common hypothesis which is used also in this model.

\begin{enh}[Multi-Country Model]
	A compromise for a future development is to model in a AB-SFC setting the main economy of the model while keeping aggregated (SFC only) the other economies.
\end{enh}

\subsection{Sectors}
The core sector of most SFC models \parencite{nikiforos2017} and a "Financial Intermediaries" sector are included.

The list is: Households ($\mathcal{H}$), Capital Goods Firms ($\mathcal{F}_K$), Consumption Goods Firms ($\mathcal{F}_C$), Financial Intermediaries ($\mathcal{I}$), Banks ($\mathcal{B}$), Government ($\mathcal{G}$), Central Bank ($\mathcal{C}$).
	
\subsection{Real Assets}
The model comprises three kind of real assets: Capital Goods ($K$) and Consumption Goods ($C$), each divided in two kinds, Essential Goods ($F$) and Luxury Goods ($L$).

\subsection{Financial Assets}
The model includes seven different financial assets. 
Bank Deposits ($D$) of Households, Firms and Financial Intermediaries. 
Banks Reserves and Government Account at the Central Bank ($A$), which are not interest-bearing. 
Central Bank Advances ($A$) to Banks.
Loans ($L$) issued by the Banks to Households and Firms. Government's Bonds ($B$) hold by Banks, Central Bank and Financial Intermediaries. 
Firms' and Banks' Equities ($E^\mathcal{F}$, $E^\mathcal{B}$) held by Financial Intermediaries.
Financial Intermediaries' Shares ($S$) held by Households.

\section{Matrices}
\subsection{Balance Sheet Matrix}
\makebox[\textwidth][c]{
\begin{tabular}{l|ccccccc|l}
	\toprule
	& $\mathcal{H}$ & $\mathcal{F}_C$ & $\mathcal{F}_K$ & $\mathcal{I}$ & $\mathcal{B}$ & $\mathcal{G}$ & $\mathcal{C}$ & Tot. \\
	\midrule
	$D$ & $+D_\mathcal{H}$ & $+D_{\mathcal{F}_C}$ & $+D_{\mathcal{F}_K}$ &  $+D_\mathcal{I}$  &  $-D$ & & & 0 \\
	$R$ & & & & & $+R_\mathcal{B}$ & $+R_\mathcal{G}$ & $-R$ & 0 \\
	$A$ & & & & & $-A$ &  & $+A$ & 0 \\
	$L$ & $-L^\mathcal{H}$ & $-L^{\mathcal{F}_C}$ & $-L^{\mathcal{F}_K}$ & & $+L$ & & & 0 \\
	$B$ & & & & $+B_\mathcal{I}$ & $+B_\mathcal{B}$ & $-B$ & $+B_\mathcal{C}$ & 0 \\
	$E$ & & $-p_E E^{\mathcal{F}_C}$ & $-p_E E^{\mathcal{F}_K}$ & $+p_E E_\mathcal{I}$ & $-p_E E^\mathcal{B}$ & $+p_E E_\mathcal{G}$ & & 0 \\
	$S$ & $+p_S S$ & & & $-p_S S$ & & & & 0 \\
	$K$ & & $+p_K K_{\mathcal{F}_C}$ & $+p_K K_{\mathcal{F}_K}$ & & & & & $+p_K K$\\
	\midrule
	Tot. & $+V_\mathcal{H}$ & $+V_{\mathcal{F}_C}$ & $+V_{\mathcal{F}_K}$ & $+V_\mathcal{I}$ & $+V_\mathcal{B}$ & $+V_\mathcal{G}$ & $+V_\mathcal{C}$ & $+p_K K$\\
	\bottomrule
\end{tabular}
}\\ \\
$V$ is the Net Worth of the sector.

\subsection{Transactions Matrix}
\makebox[\textwidth][c]{
\begin{tabular}{l|ccccccc|l}
	\toprule
	& $\mathcal{H}$ & $\mathcal{F}_C$ & $\mathcal{F}_K$ & $\mathcal{I}$ & $\mathcal{B}$ & $\mathcal{G}$ & $\mathcal{C}$ & Tot. \\
	\midrule
	Consumption & $-p_C^{\mathcal{F}} C_\mathcal{H}$ & $+p_C C$ & & & & $-p_C^\mathcal{G} C_\mathcal{G}$ & & 0 \\
	Investment & & $-p_K K_{\mathcal{F}_C}$ & $+p_K K^{\mathcal{F}_K}$ & & & & & 0 \\
	Wages & $+W$ & $-W^{\mathcal{F}_C}$ & $-W^{\mathcal{F}_K}$ & & & & & 0 \\
	Taxes & $-T$ & & & & & $+T$ & & 0 \\
	Transfers & $+M +p_C C_\mathcal{G}$ & & & & & $-M -p_C C_\mathcal{G}$ & & 0 \\
	\midrule
	$\mathcal{C}$ Profits & & & & & & $+\Pi_\mathcal{G}^\mathcal{C}$ & $-\Pi^\mathcal{C}$ & 0 \\
	$E$ Dividends & & $-\Pi^{\mathcal{F}_C}$ & $-\Pi^{\mathcal{F}_K}$ & $+\Pi_\mathcal{I}$ & $-\Pi^\mathcal{B}$ & $+\Pi_\mathcal{G}^\mathcal{F}$ & & 0 \\
	$S$ Dividends & $+\Pi_\mathcal{H}$ & & & $-\Pi^\mathcal{I}$ & & & & 0 \\
	\midrule
	$D$ Interests & $+r_D D_\mathcal{H}$ & $+r_D D_{\mathcal{F}_C}$ & $+r_D D_{\mathcal{F}_K}$ & $+r_D D_\mathcal{I}$ & $-r_D D$ & & & 0 \\
	$A$ Interests & & & & & $-r_{A} A$ & & $+r_{A} A$& 0 \\
	$L$ Interests & $-r_L L^\mathcal{H}$ & $+r_L L^{\mathcal{F}_C}$ & $-r_L L^{\mathcal{F}_K}$ & & $+r_L L$ & & & 0 \\
	$B$ Interests & & & & $+r_B B_\mathcal{I}$ & $+r_B B_\mathcal{B}$ & $-r_B B$ & $+r_B B_\mathcal{C}$ & 0 \\
	\bottomrule
\end{tabular}
}\\ \\

\section{Sectors}
\subsection{Households}
In this model the core agent (consumer, worker, capitalist) represents a household rather than a single individual. This is a very common approximation in economics and I think it is reasonable as long as we are not going into modelling education and care work, where the gender asymmetries become very relevant.

Each agent is characterized by an education level assigned when it enters the simulation replacing a retired agent inheriting their wealth, and gain experience when working in the same sector (Capital/Consumption and Essential/Luxury) without employment gap.
The education level is assigned with a probability related to the inherited wealth. 

\begin{enh}[Gender, Care work and Feminist Economics]
	Approximate individuals as household invisibilizes gender differences and the (hidden) work made mostly by women inside the family (childcare, elder-care, housekeeping, ...).
	Gender is an important factor in creating inequalities: for example unemployment and wages shows a strong gender effect (which in both cases penalizes women).
	
	Adding a gender perspective will be an improvement in the model (with relevant policy's implication) and will require to explicitly model education and childcare (which in this first draft is only sketched), the complete life-time of an agent (here reduced to the working age) and family choices (marriage, pregnancy, ...).
\end{enh}

The Households face each time the choice of what and how much to consume. 
Each Household want to consume an amount of Essential Consumption Goods randomly selected between two given quantity, with mode at the minimum (like a shifted Beta distribution). In the eventuality they are not able to satisfy their consumption desire for Essential Goods, the difference is added (discounted) to the demand of the next time.
The Demand for Luxury Good is such that it (sub-linearly) increase when the expected income increase and decrease (with a lower rate) when the income decrease. Households are willing to consume out of wealth or to ask for loans in order to satisfy their consumption desires.
If the expected capital income is sufficient to match desired consumption agents exits from the labour market (or not re-enter it).

Finally, they periodically sample agents similar to them (for sector, education, experience) and if their income is greater than their own they search for better paid job (the market is assumed rigid). 


\subsection{Firms}
Firms are characterized by their position in the supply chain (either Capital or Consumption), the supply chain in which they are insert (either Essential or Luxury) and the holder of their equities (either Government or Financial Intermediaries).

The kind of supply chain does not influence the behaviour of a firm, it simply changes the market on which the firm operates.

\subsubsection{Capital Firms}

\subsubsection{Consumption Firms}

\subsubsection{Public Sector Firms}
Public Sector Firms' Equities are held by the Government. 

They do not take loans from Banks; instead if more liquidity is required they will emit new equities which are always bought by the Government.

Both Capital and Consumption Public Sector Firms produce up to a target fixed by the Government. 
Capital Public Sector Firms sell their goods first at Consumption Public Sector Firms and then to Private Sector Firms.
Government always buys all the Consumption Public Sector Firms' production.

\subsubsection{Private Sector Firms}


\subsection{Financial Intermediaries}
The role of Financial Intermediares in the model is only to decouple the decision, made by the Households, on the amount to be financially invested and the actual choice of investment.

\subsection{Banks}
Banks fixes the interest rates on Deposits, different for each Bank but equal for each account holder.
Additionally, they chose when granting loans to other sectors (based on the balance sheet of the applicant) and fix a different interest rate for each loan.

Liquidity is obtained, in case of necessity, by selling Government's Bonds or acquiring Advances, which are always granted by the Central Bank.

Banks are required to maintain at least a given liability ($D+A$) over liquidity ($R+B$) ratio.

\subsection{Government}
Government fixes the fiscal policies, by adjusting tax rates. It determines the amount to be transfer to Households (both as monetary and non-monetary - i.e. Consumption Goods -) and the production target for Public Sector Firms.

When liquidity is needed, Government emits Bonds and sells them at will to the Central Bank. In case of excess liquidity Government can buy back Bonds from Central Bank.

\subsection{Central Bank}
In the model the role of Central Bank is to fix the Government's Bonds interest rate and the Advances interest rate.

Additionally, it passively buys and sells Government's Bonds on request and guarantees all the Advances needed by banks. Reserves and Government's Account do not grant interests.

In other words, the Central Bank is a lender of last resort for the Government, which then has no accounting limits to spending, and prevent any speculative attacks on debt.

\section{Real Assets}
\subsection{Essential Goods}
The exact definition of essential good (and service) it is not easy to be give. An intuition can be provided by the Foundational Economy approach \parencite{arcidiacono2018a}: \begin{quote}
	The sphere of the foundational was then demarcated by three criteria: these goods and services were necessary to everyday life; were consumed daily by all citizens regardless of income; and were distributed according to population through branches and networks. They were partly non-market, generally sheltered and one way or another politically franchised.
\end{quote}

Operationally, we can image the essential goods in the model as the ones included in the basket used by national statistics offices to determinate the poverty line. In this sense, it is a set of goods which continuously mutate to adapt to new life needs.

\begin{enh}[Housing]
	Among essential goods one should require ad hoc modelling: houses. Houses are special for three reasons.
	
	First, they are very heterogeneous in prices and quality, and both are strongly related to the position. In other words, including houses requires (quite always) to make the model spatially explicit.
	
	Second, the expenses for housing, in form of rent or mortgage, account for a significant part of monthly consumptions for poor individuals (up to one half).
	
	Third, real estate properties are an important form of rent extraction and an important tool of investment, and so another important channel of redistribution.
\end{enh}

\subsection{Luxury Goods}
Luxury goods are, by exclusion, all the non-essential goods.

\begin{enh}[Diversified Goods]
	A subsequent version of the model can include different (abstract) goods to be produced and consumed. This will  create two different innovation processes (better technology for existing goods, or technology for new goods) and will account for the empirical fact that higher the income more diversified the consumptions are \parencite[cfr.][§2]{didomenico2022}.
\end{enh}

\subsection{Capital Goods}

\section{Financial Assets}

\subsection{Advances}
Advances are provided upon request by the Central Bank and do not expire. Each period, interests are paid according to the rate fixed by the Central Bank.

\subsection{Loans}

\subsection{Bonds}
Bonds are sold and bought at their nominal value and do not expire. Central Bank satisfies every transaction. Each period, interests are paid according to the rate fixed by the Central Bank.

\subsection{Equities}

\subsection{Shares}
Financial Intermediaries' Shares are unlimited in number and re-bought by the Financial Intermediaries at their nominal value on request.

\section{Model steps}
\begin{steps}
	\item \label{C1} Every Q times, Central Bank updates interest rates []
	\item \label{G1} Every T times, Government update tax policies []
	\item \label{G2} Government set target output for Public Sector Consumption Firms []
	\item \label{F1} Consumption Firms set desired output, acquire Loans from Banks [\cref{C1}]
	\item \label{F2} Consumption Firms open job vacancies, order Capital goods to Capital Firms [\cref{G2, , F1}]
	\item \label{F3} Capital Firms set desired output, acquire Loans from Banks [\cref{C1, ,F2}]
	\item \label{F4} Capital Firms open job vacancies [\cref{F3}]
	\item \label{H1} Households set demand for Consumption Goods [\cref{C1, ,F2, ,F4}]
	
\end{steps}

\section{Equations}

\section{Parameters}
	
	\printbibliography
	
\end{document}
